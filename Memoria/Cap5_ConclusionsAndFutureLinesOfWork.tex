% ---------------------------------------------------
%
% Trabajo de Fin de Grado. 
% Author: Alejandro Hernández Padrón. 
% Capítulo: Conclusiones y lineas de trabajo futuras. 
% Fichero: Cap5_ConclusionsAndFutureLinesOfWork.tex
%
% ----------------------------------------------------
%

\chapter{Conclusiones y líneas de trabajo futuras} \label{chap:Conclusiones} 

En este capítulo se presentarán las conclusiones a las que se ha llegado tras realizar este TFG y discutiremos posibles líneas de trabajo futuras.

\subsection{Conclusiones}

Actualmente la tecnología beacon se encuentra aún en fase de desarrollo. Por sí sola puede llegar a tener muchas limitaciones debido a su funcionamiento. Las restricciones físicas y de posicionamiento requieren que se realice un análisis para posicionar los dispositivos de la mejor manera. Por otro lado cada distribuidor específico de beacons intenta que se utilicen sus SDKs para desarrollar las aplicaciones. Dependiendo del caso, incluso existen algunos sujetos a cuotas, con lo que desarrollar utilizando esta tecnología puede no resultar económico y poner las cosas difíciles a muchos desarrolladores. La solución a este problema la hemos encontrado en la librería AltBeacon que intenta abrir las puertas a los desarrolladores que pretenden utilizar esta tecnología.


Gracias a esta librería se están desarrollando diferentes aplicaciones, que podrían llegar a tener funcionalidades interesantes para el día a día como la automatización de algunas tareas. Hasta ahora se han explorado las posibilidades más beneficiosas a la hora de vender la tecnología como es la publicidad, pero al desarrollar este trabajo, hemos comprobado que el mercado es mucho más amplio, abarca mucho más de lo que se ha desarrollado hasta ahora. 


La consecución de este TFG muestra que existen muchas aplicaciones que podrían interactuar con esta tecnología y dar lugar a un producto útil y cómodo para los usuarios. En unos años probablemente seamos capaces de explotar al máximo las posibilidades que nos ofrecen los beacons, dando paso a aplicaciones que formen parte de nuestra rutina, sin embargo por ahora seguiremos experimentando con esta tecnología y viendo como evoluciona.


\subsection{Conclusions}

Nowadays, beacon technology is still on a development phase. By itself can be quite limitated due to to its functioning.The physical and positional limitations require a deep analysis in order to place devices the best way. On the other hand, each specific beacon provider tries to use his own SDKs to develop the aplications. Depending on the case, fees might be applied so, working with this technology can end up being not money-saving and make things difficult for many developers.


The solution to this problem has been found on the AltBeacon library, which tries to bring close to developers this technology. Thanks to this library, many applications, which might have handy functionalities for day-to-day usage, have been developed and many related to the automation of tasks. Until now just the advertising possibilites have been explored, since it is the most profitable option. 


However, during this project we have realized that market is far more wide and it covers much more than what has been developed right now. The achievement of this Final Year Project is to show that, many apps able to work with this technology exist and they can give as a result an useful and convenient product for users. 


In a few years, we will probably be able to take advantage of all posibilities beacons offer and we will be able to develop apps, which would end up beeing part of our routine; for now, we will keep on testing this technology and watching it grow.

\subsection{Líneas de trabajo futuras}


En un futuro se podrían ampliar las funcionalidades desarrollando los demás casos de uso planteados que no se han elegido en este trabajo. En entornos universitarios existen multitud de servicios que podemos relacionar con esta tecnología, entre otros podemos destacar: 


\begin{itemize}
\item Acceso a instalaciones deportivas, educativas o de diversa índole.
\item Contratación de servicios en diferentes puntos: alquiler de bicicletas en puntos de alquiler, pago de entradas en las inmediaciones de un evento.
\item Anuncios o promociones de diverso tipo, relacionados tanto con la universidad como con establecimientos comerciales cercanos al campus.
\item Servicios de guía dentro del campus, con puntos de interés para el alumnado.
\end{itemize}


Son múltiples las posibilidades que se le pueden dar a esta tecnología. En otros sectores como el turismo o la medicina también están empezando a tener relevancia, con lo que podemos confirmar que no es una tecnología aislada, sino que poco a poco se está abriendo paso en el mercado.









