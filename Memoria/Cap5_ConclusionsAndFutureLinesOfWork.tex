% ---------------------------------------------------
%
% Trabajo de Fin de Grado. 
% Author: Alejandro Hernández Padrón. 
% Capítulo: Conclusiones y lineas de trabajo futuras. 
% Fichero: Cap5_ConclusionsAndFutureLinesOfWork.tex
%
% ----------------------------------------------------
%

\chapter{Conclusiones y líneas de trabajo futuras} \label{chap:Conclusiones} 

En este capítulo se presentarán las conclusiones a las que se ha llegado tras realizar este TFG y discutiremos posibles líneas de trabajo futuras.

\subsection{Conclusiones}


\subsection{Conclusions}

Nowadays, beacon technology is still on a development phase. By itself can be quite limitated due to to its functioning.The physical and positional limitations require a deep analysis in order to place devices the best way. On the other hand, each specific beacon provider tries to use his own SDKs to develop the aplications. Depending on the case, fees might be applied so, working with this technology can end up being not money-saving and make things difficult for many developers.


\subsection{Líneas de trabajo futuras}



\begin{itemize}
\item Acceso a instalaciones deportivas, educativas o de diversa índole.
\item Contratación de servicios en diferentes puntos: alquiler de bicicletas en puntos de alquiler, pago de entradas en las inmediaciones de un evento.
\item Anuncios o promociones de diverso tipo, relacionados tanto con la universidad como con establecimientos comerciales cercanos al campus.
\item Servicios de guía dentro del campus, con puntos de interés para el alumnado.
\end{itemize}


Son múltiples las posibilidades que se le pueden dar a esta tecnología. En otros sectores como el turismo o la medicina también están empezando a tener relevancia, con lo que podemos confirmar que no es una tecnología aislada, sino que poco a poco se está abriendo paso en el mercado.









