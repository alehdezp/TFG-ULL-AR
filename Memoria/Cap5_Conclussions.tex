% ---------------------------------------------------
%
% Trabajo de Fin de Grado. 
% Author: Alejandro Hernández Padrón. 
% Capítulo: La aplicación ULL-AR. 
% Fichero: Cap5_StartApplication.tex
%
% ----------------------------------------------------
%

\chapter{Conclusiones y futuras líneas de Trabajo} \label{chap:Conclusiones} 

En este capítulo se presentarán las conclusiones a las que se ha llegado tras realizar este TFG y se discutirán posibles líneas de trabajo futuras.

\section{Conclusiones}
 
Actualmente la realidad aumentada está invadiendo nuestras vidas, cada día se descubren nuevas muestras de lo que esta tecnología es capaz. La gran mayoría de sus futuras aplicaciones se encuentran aún en desarrollo debido a lo reciente y compleja que es esta tecnología. La RA se encontrará integrada en múltiples ámbitos tanto como profesionales dados sus aplicaciones en medicina e ingeniería, como en el de la enseñanza permitiendo a los estudiantes mejorar e incentivar el proceso de aprendizaje. 

En cuanto a la implementación y desarrollo de aplicaciones de RA, se encuentran limitadas al avance en el desarrollo de los SDK de RA que ofrecen empresas como Google, Apple o Vuforia, que cada día van ofreciendo más funcionalidades y mejor rendimiento pero que todavía tienen un amplio margen de mejora en función de lo que se espera de esta tecnología. Uno de los principales inconvenientes que se encuentra a la hora de desarrollar una aplicación de RA es elegir un SDK de RA que se adapte tanto a los requisitos como al presupuesto, ya que aunque la mayoría de los proveedores de estos SDK tienen versiones gratis para desarrolladores, a la hora de querer integrar esta tecnología en una aplicación,  la mayor parte de sus funcionalidades más interesantes para el reconocimiento de objetos en la nube, se encuentran con unos precios mensuales fuera del alcance de muchas empresas y desarrolladores.

En cuanto de la integración de un SDK de RA en Android Studio, se han encontrado con diversos problemas que dificultan trabajar con la RA en este IDE debido a que la mayoría de las empresas de RA que se encargan de desarrollar una versión de sus SDK para Android Studio, dejan de lado muchas funcionalidades y no ofrecen la documentación suficiente para instalar y empezar a trabajar con el SDK. En caso de querer realizar una aplicación de realidad aumentada de cierta calidad se recomienda el uso de plataformas de desarrollo 3D como Unity.



\section{Conclusions}

Nowadays, augmented reality is invading our lives, every day new samples of what this technology is capable of are being discovered. The great majority of its future applications are still under development due to the recent and complex nature of this technology. AR will be integrated in multiple fields, both as professionals given its applications in medicine and engineering, as well as in teaching, allowing students to improve and encourage the learning process. 

As for the implementation and development of AR applications, they are limited to the advance in the development of AR SDKs offered by companies such as Google, Apple or Vuforia, which every day are offering more functionality and better performance but still have a wide margin for improvement depending on what is expected of this technology. One of the main drawbacks when developing an AR application is to choose an AR SDK that fits both requirements and budget, since although most of the providers of these SDK have free versions for developers, when trying to integrate this technology into an application, most of its most interesting features for the recognition of objects in the cloud, are with monthly prices beyond the reach of many companies and developers.

As for the integration of an AR SDK in Android Studio, they have encountered several problems that make it difficult to work with AR in this IDE because most of the AR companies that are responsible for developing a version of their SDK for Android Studio, leave aside many features and do not offer enough documentation to install and start working with the SDK. If you want to make an augmented reality application of a certain quality is recommended the use of 3D development platforms such as Unity.

% \section{Líneas de trabajo futuras}

% En un futuro se podría ampliar la funcionalidad de la aplicación permitiendo a los usuarios no solo llegar a una instalación, si no moverse dentro de ellas para llegar a una clase, un laboratorio, el despacho de un profesor, etc. Sería de gran ayuda el uso de la tecnología de los ``beacons'' \cite{URL::beacon} para ubicar al dispositivo móvil dentro de una instalación.

% A su vez se podría implementar un algoritmo más preciso para la identificación de las instalaciones, ya que en las instalaciones de gran superficie el algoritmo utilizado no es el más optimo debido a que para el cálculo solo se tiene en cuenta un punto geográfico que encuentra en el centro de cada instalación. Lo ideal sería tener un área generada con los puntos geográficos de cada instalación e identificarlas cuando el usuario enfoque a esa área.




% El principal inconveniente con el que se encuentra la RA en el ambito profesional y de la enseñanza que los dispositivos  con suficiente capacidad de cómputo y que a su vez sean un medio hábil y práctico para su visualización, como pueden ser las Google Glass no se encuentran al alcance de cualquiera debido a su

