
\chapter{Presupuesto} \label{chap:RAEntornosUniversitarios} 

En este capítulo se expondrán las estimaciones de recursos necesarios para
desarrollar y publicar el proyecto.

El desglose del presupuesto de la aplicación se puede contabilizar de la siguiente manera:

\begin{itemize}
    \item En caso de que se desee publicar la aplicación en la Google Play Store, se necesita poseer de una cuenta de desarrollador de Google. Para ello se necesitará disponer de una cuenta de Google a una de desarrollador, se han de seguir los pasos en la web \cite{URL::googleplayconsole}, en la que pedirán un abono único de 25 dólares.  
    \item En cuanto a coste del servidor, Heroku ofrece distintos planes para la gestión de este.  En Heroku los ``dynos'' son las unidades que proveen capacidad de cómputo. El pago mensual en Heroku dependerá de la cantidad y la potencia de los dynos que se tengan contratados. Las cuotas por dyno que ofrece Heroku son:
    \begin{itemize}
        \item Free: 0 dólares al mes. 
        \item Hobby: 7 dólares por dyno al mes.
        \item Standard 1x: 25 dólares por dyno al mes.
        \item Standard 2x: 50 dólares por dyno al mes.
        \item Perfomance M: 250 dólares por dyno al mes.
        \item Perfomance L: 500 dólares por dyno al mes.
    \end{itemize}
    
    Los planes Free y Hobby no constan de mucha potencia y tienen limitados el número de procesos que pueden responder a las peticiones, son recomendadas para uso personal y de pequeñas aplicaciones. El resto de los planes ofrecen número de procesos ilimitados y permite la escalabilidad horizontal en el caso de que la aplicación lo requiera.

    Para los requisitos de la aplicación se necesitaría contratar  un dyno del plan Standard 1x. El coste mensual de este plan sería de 300 dólares al año.
    En caso de que el dyno contratado no sea capaz de soportar el tráfico de la aplicación se podría cambiar de plan o agregar otro dyno Standard 1x. 

    \item La base de datos se encuentra situada en la plataforma de mLab. Esta plataforma en función del número de GB que se necesiten para la base de datos ofrece los siguientes planes:
    \begin{itemize}
        \item Sandbox: 0.5 GB gratis. 
        \item Shared Cluster: 15 dólares al mes por GB.
        \item Dedicated Cluster: a partir de 180 dólares al mes por 40 GB. 
    \end{itemize} 

    El plan de Sandbox se ajusta a los requisitos de la base de datos de la aplicación, por lo que no tendría coste. 
    
    \item La información referente a cada instalación sería recomendable obtenerla de primera mano del personal encargado de cada instalación para asegurar que la información este completa y correcta. Esto supondría un costo de 300 o 400 euros para gastos de transporte e investigación.
    \item  El salario promedio de los programadores en España es de unos 1600 euros al mes, de modo que el
		coste por hora se puede estimar en unos 10 euros por hora de trabajo (teniendo en cuenta 8 horas de
		trabajo diarias, 40 semanales y 160 mensuales).
		Para poder desarrollar la aplicación sería necesario contratar a un programador experto Android.
	Teniendo en cuenta que el desarrollo de \ULLAR{} se estima en unos dos meses de trabajo en estas
	condiciones, ello totalizaría unos 3200 euros por el desarrollo de la aplicación. 
    \item Sería muy conveniente el mantenimiento de la aplicación. 
		Se ofrece un programador en Android con conocimientos en Node.js que se haría cargo de las incidencias que pudieran surgir, posibles mejoras puntuales de la aplicación,
		del mantenimiento del servidor y de actualizar la información de la base de datos. 
		Este programador trabajaría con un coste por hora de 20 euros/hora, de modo que el coste del mantenimiento
		se calcularía por parte de la organización interesada en \ULLAR{} mediante la contratación de una ``bolsa
		de horas'' de trabajo de este programador.
\end{itemize}  

El cálculo del coste total de la aplicación se hará en función de los puntos comentados anteriormente.

