%
% ---------------------------------------------------
%
% Proyecto de Final de Carrera:
% Author: Alejandro Hernández Padrón <alu0100703511@ull.edu.es>
% Capítulo: Objetivos 
% Fichero: Cap1_Goals.tex
%
% ----------------------------------------------------
%

\chapter{RA en entornos universitarios } \label{chap:RAEntornosUniversitarios}  

En este capítulo trataremos los usos y ventajas de la integración de la realidad aumentada en entornos universitarios.

La realidad aumentada se presenta en el ambito educativo como una tecnología capaz de aportar transformaciones significativas en la forma en que los estudiantes perciben y acceden a la realidad física, proporcionando así experiencias de aprendizaje más ricas e inmersivas.

Con la continua implatación de nuevas tecnologías en las aulas, junto al incremento de dispositivos móviles en la población, sitúa a la RA en una posición destacada en la actualidad. En la actualidad en educación, la realidad aumentada rara vez se usa, pero cada vez más docentes, investigadores y desarrolladores están comenzando a moverse hacia nuevos métodos de enseñanza más interactivos.

% La tecnología de la RA tiene la capacidad de cambiar la forma en la que los estudiantes perciben la realidad física, puesto que permite ampliar el conjunto de información que obtenemos de ella, para facilitar la captación de sus características y elementos.

La tecnología de la RA permite cambiar la forma de entender los contenidos de aprendizaje, puesto que aporta nuevas formas de interacción con el mundo real a través de capas digitales de información que amplían, completan y transforman en cierto modo la información inicial.

La aplicaciones que tiene la RA, en lo referente a la creación de materiales didácticos y actividades de aprendizaje son múltiples, directas y fáciles de imaginar en prácticamente todas las disciplinas, sobre todo, las relacionadas con las ciencias aplicadas (ingeniería, química y física, biología), pero también en el campo del diseño industrial, la cirugía, la arqueología, etc.


Los beneficios potenciales de la RA aplicados a la educación incluyen:
\begin{itemize}

    \item Aumentar o enriquecer la información de la realidad para hacerla más comprensible al estudiante.
    
    \item Permite múltiples formas de visualización de conceptos teóricos difíciles.

    \item El uso de una interfaz tangible para la manipulación de objetos, que permite observar un objeto desde diferentes puntos de vista,
    seleccionando, el estudiante, el momento y posición de observación.

    \item Potenciar el aprendizaje ubicuo \cite{URL::AprendizajeUbicuo}.
    
    
    \item Crear escenarios “artificiales” seguros para los estudiantes
    como pueden ser laboratorios o simuladores. 

    \item Enriquecer los materiales impresos para los estudiantes con información adicional en diferentes soportes.
    
    \item Facilita la colaboración efectiva y discusión entre los estudiantes.
\end{itemize}

Hay que tener en cuenta una serie de requisitos a la hora de diseñar un sistema educativo de RA. Estos requisitos son:

\begin{itemize}
    \item Ser sencillo y robusto.
    \item Permitir que el educador ingrese información de manera simple y efectiva.
    \item Proporcionar al alumno información clara y concisa.
    \item Permitir una fácil interacción entre estudiantes y educadores.
    \item Realizar procedimientos complejos transparentes para los alumnos.
    \item Ser rentable y fácilmente extensible.
\end{itemize}


\section{Posibles aplicaciones}

\subsection{Prácticas en laboratorios} 
Los laboratorios, poseen instrumental de aprendizaje que engloba más información de la que por su apariencia aporta, lo que hace que sea un escenario ideal para el uso de tecnología como la realidad aumentada. A todos aquellos elementos que podemos encontrar en un laboratorio se les puede asociar video tutoriales con información de uso, información en texto, archivos audibles, etc., que pueden ser accesibles de forma sencilla desde un dispositivo móvil. Permite la realización de prácticas en las que el profesor añada la información a los elementos del laboratorio y los alumnos sean los que consulten esa información, pueden ser los propios alumnos los que integren la información en el laboratorio, pueden crear varios puestos con información a modo de instrucciones de una práctica, etc.

\subsection{Trabajos de campo} 
Al igual que en el caso de los laboratorios cualquier experiencia o práctica que hagamos es susceptible del uso de la realidad aumentada. Se podrá asociar información a un entorno objeto de estudio tanto por parte del alumnado como el profesorado para su trabajo de forma experimental de una forma muy sencilla. De esta manera, objeto de conocimiento y conocimiento se dan en el mismo tiempo y lugar. Un par de ejemplos pueden ser la realización de rutas por ciudad visitando lugares emblemáticos y descubriendo la información asociada a esos sitios, estatuas, edificios, monumentos, etc., o por zonas rurales, de montaña en las que podríamos identificar especies, accidentes geográficos, etc.
    
\subsection{Eventos}  
En este tipo de ejemplo de uso cabrían las exposiciones, seminarios, jornadas, encuentros, etc. A través de la documentación que se realiza para los asistentes, ponentes y a modo de publicidad se pueden incluir códigos QR en posters informativos, en folletos, catálogos o en las webs de los eventos. Si utilizamos una aplicación específica de igual manera puede incluirse información adicional. Es un recurso muy interesante ya que es un modo de incluir gran cantidad de información asociada al evento accesible con cualquier soporte móvil en cualquier sitio y lugar debido a la ubicuidad de estos dispositivos.
    
\subsection{Libros}
A los libros electrónicos o en formato papel se añade realidad aumentada utilizando como activador de la información los textos, ilustraciones, encabezados, pies de página, etc., y como información adicional en muchos casos se incluye la biografía del autor, los pies de página, vídeos que desarrollan la acción más ampliada, textos adicionales y audios. Se denominan libros aumentados.
o El libro enmarcado en el proyecto: “HUSSO DIGITAL: LA CIUDAD UNIVERSITARIA EN REALIDAD AUMENTADA, “El libro aumentado de Eduardo Torroja” es un claro ejemplo.

\subsection{Visitas}
En muchos casos, a lo largo del curso académico se realizan salidas fuera del aula y se visitan lugares como complemento educativo a las clases regladas. Los museos, galerías, fábricas, empresas, incorporan la realidad aumentada en sus recorridos proporcionando una información completa y audiovisualmente muy atractiva a los visitantes. Los estudiantes además de aprender la materia objeto de la visita desarrollan las destrezas que el manejo de esta tecnología les proporciona.

\subsection{Aprendizajes experimentales} 
prácticamente todas las disciplinas tienen una parte experimental que pueden realizarse con realidad aumentada facilitando en gran medida el aprendizaje y el desarrollo de destrezas transversales. Ejemplos claros pueden ser en medicina, donde el uso de las google glass de forma experimental hace un par de años fue muy mediático, en arquitectura e ingenierías, la posibilidad de realizar y ver modelos en 3D de diferentes edificios y construcciones es muy útil en el aprendizaje del alumno. En química o física con aplicaciones como las que aparece en el bloque 3 dedicado a programas y aplicaciones, también en ramas como la Gabinete de Tele-Educación 24 Universidad Politécnica de Madrid Realidad Aumentada en Educación biología, arte, historia, diseño, idiomas, geografía, matemáticas, urbanismo, música, geometría, etc.


\subsection{ejemplos}

Desde la Escuela Técnica Superior de Ingenieros en Topografía, Geodesia y Cartografía de la Universidad Politécnica de Madrid y de una manera muy sencilla nos presentan una práctica extrapolable a cualquier disciplina interesada en la observación del relieve de los mapas. Consiste en un simulador con un cajón de arena moldeable y tras ser capturado por una cámara con un sensor infrarrojo calcula un modelo digital tridimensional del terreno, además se proyectan sobre la arena las curvas de nivel, coloreadas con tintas hipsométricas. Permite la simulación de agua y de volcanes a modo de realidad aumentada. Sobre la arena se verá la representación cartográfica.

http://idav.ucdavis.edu/~okreylos/ResDev/SARndbox/



Varios científicos de la Universidad Carlos III de Madrid han diseñado unas gafas inteligentes que permiten conectar a profesores y alumnos en tiempo real en el aula. Telmo Zarraonandia, Ignacio Aedo, Paloma Díaz y Álvaro Montero a través del artículo, An augmented lecture feedback system to support learner and teacher communication explican su funcionamiento. Con tan solo ponerse las gafas en cuestión, el docente obtendrá información del alumno al mirar tras ellas. Notas y comentarios que lanzarán los alumnos al docente los podrá recibir con tan solo observar a su grupo de clase.

El feedback estará asegurado en estas aulas, de nuevo utilizando como recurso tecnológico la realidad aumentada.

% https://e-archivo.uc3m.es/bitstream/handle/10016/17136/augmented_british_2013_pp.pdf?sequence=1&isAllowed=y




Collaborative work of students within the
Augmented Reality application Construct3D

Libros de texto aumentados 



Modelos 3D utilizados en arquitectura para la visualización de dispositivos







Casos de usos. 
 
\section{Aplicaciones móviles en entornos universitarios}


\begin{itemize}
\item XXX
\item XXX
\end{itemize}



\subsection{Descarga automática de material} \label{sec:descargaautomatica}




