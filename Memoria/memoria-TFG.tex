\documentclass{article}

\title{TFG-ULL-NAVIGATION}
\date{2018-04-08}
\author{Alejandro Hernandez Padron}

\begin{document}
    \pagenumbering{gobble}
    \maketitle
    \newpage
    
    \section{Resumen}

    En este proyecto se propone realizar una aplicación para dispositivos móviles Android enfocada a entornos universitarios, utilizando como plataforma de desarrollo el IDE de Android Studio. La aplicación se personalizará para su utilización, a nivel de prototipo en la Universidad de La Laguna.\\

    Con este proyecto el alumno se dispone a profundizar en sus conocimientos sobre en lenguaje de programación Java, así como en tecnologías móviles en el sistema operativo Android y familiarizarse con el uso de diferentes APIs de Google.
    
    \newpage
    \section{Objetivos}
    La Universidad de La Laguna cada vez se va actualizando más hacia nuevas tecnologías. Sin embargo en el sector de los móviles, la ULL carece de presencia y debido a esto el alumno se propone realizar una APP que cuente con algunos servicios básicos orientados a el alumnado de esta universidad.\\

    Este proyecto se dispone a realizar una aplicación de Android con los siguientes objetivos principales:

    \begin{itemize}
        \item  	Permitir al usuario navegar mediante los sistemas de posicionamiento de Google, por los distintos campus, facultades y servicios de la ULL, ofreciendo al usuario una información general de cada una de las instalaciones universitarias disponibles a su alrededor.
        \item   Diseñar un prototipo de aplicación móvil que haga uso de técnicas de realidad aumentada para suministrar al usuario de la aplicación, información sobre instalaciones universitarias.
        \item   La implementación de un sistema de gestión de incidencias, relacionadas con algún tipo de inconveniente o problema relacionados con alguna instalación universitaria ULL, como por ejemplo problemas de accesibilidad con personas con discapacidad, problemas con el material del aula, etc.
    \end{itemize}

    \newpage
    \section{Herramientas y Tecnologias}
    A continuación se explicarán brevemente las distintas herramientas software utilizadas en el proyecto. 
    \subsection{Android Studio}
    Android Studio es el IDE (Entorno de Desarrollo Integrado) oficial para el desarrollo de aplicaciones en Android, basado en IntelliJ IDEA. Android Studio ofrece una serie de funcionalidades que han facilitado a la desarrolladora numerosas tareas, entre las cuales podemos destacar:
    \begin{itemize}
    \item Un sistema de compilación basado en Gradle que ha simplificado tanto la inserción de dependencias de las distintas librerías que se han tenido que utilizar, como la compilación de la aplicación.
    \item Un emulador rápido y fácil de utilizar que ha ayudado a visualizar las distintas pantallas durante el desarrollo aunque no ha sido de mucha utilidad para probar el funcionamiento al ser dependiente la app de la tecnología Bluetooth.
    \item La facilidad para publicar cambios a aplicaciones ya funcionando sin tener que eliminar y volver a crear un nuevo APK parando la app.
    \item Un sistema de visualización de las diferentes pantallas muy completo, con soporte visual para añadir componentes y cambiar atributos fácilmente.
    \item Un sistema de depuración, con una interfaz sencilla e intuitiva.
    \end{itemize} 

    Se ha utilizado este IDE frente a otros como Eclipse + ADT debido a que en la actualidad es el IDE oficial con soporte de Google. Se ha preferido aprender a utilizar este entorno con vistas al futuro, ya que parece que se consolidará como el preferido para los desarrolladores Android.

    \subsection{LaTex}

LaTeX es un sistema de composición de textos, orientado a la creación de documentos que presenten una alta calidad tipográfica. Por sus características y posibilidades, es usado especialmente en la generación de artículos y publicaciones científicas que incluyen, entre otros elementos, expresiones matemáticas, gráficos o figuras.


LaTeX está formado por un gran conjunto de macros de TeX, escrito por Leslie Lamport en 1984, con la intención de facilitar el uso del lenguaje de composición tipográfica, creado por Donald Knuth. LaTeX es software libre bajo licencia LPPL.


Se ha decidido utilizar este sistema debido al carácter profesional que aporta a los documentos. Ha sido una buena oportunidad para aprender a usar un sistema de composición de texto como este, ya que en un futuro puede ser beneficioso el saber manejar esta herramienta. 


Si bien es cierto, que el uso de esta herramienta frente a otros editores más familiares ha sido algo tedioso en el inicio, es verdad que una vez acostumbrada a su uso ha resultado ser muy eficaz. En el proceso de aprendizaje se recurrió principalmente a manuales por internet, alguno a destacar en español sería

\subsection{Github}

GitHub es una forja (plataforma de desarrollo colaborativo) para alojar proyectos que utiliza el sistema de control de versiones Git. Utiliza el framework Ruby on Rails por GitHub, Inc. (anteriormente conocida como Logical Awesome). Desde enero de 2010, GitHub opera bajo el nombre de GitHub, Inc. El código se almacena de forma pública, aunque también se puede hacer de forma privada, creando una cuenta de pago.


Se ha decidido crear un repositorio en esta plataforma para poder llevar un control y una trazabilidad del proyecto. El tutor y la alumna han trabajado en este repositorio de manera conjunta. En el caso del tutor, principalmente para revisar el seguimiento semanal y llevar un control de las tareas. En el caso de la alumna, para tener un repositorio donde subir los distintos elementos que se han ido generando a lo largo del trabajo. Aparte de este repositorio, también se ha abierto un segundo repositorio asociado a la oficina del software libre (OSL) para subir el código una vez terminado como parte del programa de apoyo a trabajos finales libres (PATFL) de la ULL.


Mediante el uso de este repositorio, la alumna ha conseguido ampliar sus conocimientos en Git y familiarizarse con la interfaz de GitHub. Previamente se había utilizado como repositorios GitLab, SVN y RTC en otros proyectos, por lo que no ha sido una complicación mayor utilizar este sistema.

\section{Tecnologías utilizadas}

A continuación se revisan las distintas tecnologías utilizadas en el desarrollo de la aplicación.

\subsection{El Sistema Operativo Android}

Android es un sistema operativo que emplea Linux en la interfaz del hardware.  Los componentes del SO subyacentes se codifican en C o C++ pero las aplicaciones se desarrollan en Java. De esta manera Android asegura una amplia operatividad en una gran variedad de dispositivos debido a dos hechos: la interfaz en Linux ofrece gran potencia y funcionalidad para aprovechar el hardware, mientras que el desarrollo de las aplicaciones en Java permite que Android sea accesible para un gran número de programadores conocedores del código.

Este SO fue diseñado principalmente para dispositivos móviles con pantalla táctil: smartphones, tablets y otros dispositivos como televisores o automóviles. Fue desarrollado inicialmente por Android Inc., empresa que fue respaldada económicamente por Google y más tarde adquirida por esta misma empresa.

Actualmente tiene una gran comunidad de desarrolladores creando aplicaciones para extender la funcionalidad de los dispositivos. A fecha de hoy existen más de un millón de aplicaciones disponibles para la tienda oficial de Apps de Android, Google Play sin tener en cuenta las aplicaciones de otras tiendas no oficiales, como por ejemplo, la tienda de aplicaciones de Samsung Apps.

    
\end{document}