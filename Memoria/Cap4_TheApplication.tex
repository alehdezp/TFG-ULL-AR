% ---------------------------------------------------
%
% Trabajo de Fin de Grado. 
% Author: Alejandro Hernández Padrón. 
% Capítulo: La aplicacion BulletPoint. 
% Fichero: Cap4_TheApplication.tex
%
% ----------------------------------------------------
%

\chapter{La aplicación BulletPoint} \label{chap:LaAplicacion} 

Basándonos en los casos de uso revisados en el capítulo anterior, en este capítulo se discutirán los casos de uso que han sido elegidos e implementados en la aplicación \BulletPoint{}. Comentaremos la aplicación centrándonos en el desarrollo de la misma, así como  en diferentes partes a destacar del código que puedan resultar interesantes.


\section{Casos de uso elegidos}

\subsection{Localización de transporte público, horarios e \\información de la parada}

\subsubsection{Objetivo}


El objetivo de este caso de uso es el conocer, en tiempo real, qué autobuses pasan por la parada en la que nos encontramos, hacia dónde se dirigen, y cuánto tiempo falta para que lleguen a la parada. En caso de necesidad de información adicional, la aplicación está preparada para remitirnos a navegar por la página web \cite{URL::titsa} de la empresa Transportes interurbanos de Tenerife, S.A.(TITSA), donde podemos ver datos adicionales sobre el autobús seleccionado.


\lstinputlisting[float, floatplacement=H, caption={La clase \textit{Arrival} donde quedan contenidos los datos de cada llegada.}, label={code:arrival}]
{listings/Arrival.java} %% LISTING

En un principio nuestra idea era utilizar simplemente la API de TITSA, puesto que se creía que proporcionaría toda esta información; sin embargo, en algunos casos los destinos no se correspondían con la realidad y las rutas aparecían erróneas. Ante esta situación, se contactó con personal de  TITSA involucrado en el desarrollo de esta API quien nos comentó que esto ocurría en algunos casos por la manera en la que estaba planteada la API.


En el primer método, utilizando una librería de peticiones HTTP, se envía una consulta por GET a la API de TITSA que devuelve una respuesta en XML. Esta respuesta se parsea con un handler de XML (véase Listado \ref{code:handler}). 
\vspace{5mm}



En cuanto se accede al módulo de parking de \BulletPoint{}, la aplicación muestra una lista con los diferentes aparcamientos y el número de plazas de cada recinto. Esta lista se obtiene mediante una consulta a una API diseñada en colaboración con Alberto Morales, quien se ha encargado de configurar el servicio para facilitar a la aplicación la información necesaria. De la lista de aparcamientos, el usuario ha de seleccionar al que quiere acceder, identificando así la imagen y zonas a dibujar. Esta selección da paso a un escáner que comienza a detectar la posición del usuario y a situarlo en la imagen. Si el usuario se encuentra en una de las zonas designadas para entrar o salir del parking, se lanzará la acción, que en este caso consiste en una petición de apertura contra el servidor. En este caso cada zona posee un identificador de barrera que es el que usa la petición para determinar que barrera debe abrir. 


Todos estos factores han de ser tomados en cuenta a la hora de realizar el despliegue de los beacons. Otro factor a tener en cuenta es la altura. Los dispositivos es recomendables levantarlos cierta altura sobre el nivel del suelo. A la hora de probarlos se ha intentado, en la medida de lo posible, mantenerlos a un nivel elevado sobre la altura de las personas y siempre intentando mantener esta medida de altura para los diferentes beacons. En cuanto al despliegue de la aplicación, el código fuente de \BulletPoint{} se encuentra disponible para su descarga en \cite{URL::repositorioAplicacion}, bajo licencia Creative Commons Reconocimiento-NoComercial-CompartirIgual 4.0 Internacional \cite{URL::licencia}.





